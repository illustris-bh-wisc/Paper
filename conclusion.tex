\section{Conclusions}

\label{sec:conclusions}

In this work, we have analyzed the complete set
of supermassive black holes from the state-of-the-art cosmological
hydrodynamical simulation Illustris-2. Applying well-known models
of black hole phenomena, we calculated the X-ray luminosity 
as a function of black hole mass and accretion rate. 

We reproduced the fundamental plane of black hole activity with the
Illustris data sample and showed that when modeled with the thin-disk
accretion approximation, the efficiency coefficient $q$ (the slope of
a secondary parametrization of the M03 fundamental plane) obtained is within
a factor of two of the value obtained in that work. On the other hand, when modeled with the Eddington
accretion approximation, we found that the model gives a non-physical
value for the efficiency of the accretion mechanism. This signifies
that the Eddington accretion mechanism does not model well the accretion of
black holes with the data in the Illustris simulation.

In future work, we plan to perform the following analyses:
\begin{itemize}
\item Go to higher redshift to trace the evolution of the fundamental plane
over cosmic time
\item Test the Bondi-Hoyle-Lyttleton model for non-spherical accretion
\item Trace the temperature and density of the gas inside of the halos as
a function of redshift to determine how the feedback is affecting
the environment
\item With the temperatures, we can then estimate radio luminosities to
more exactly compare with the M03 fundamental plane
\item Examine the effect of mergers on the slope of the fundamental plane
\end{itemize}

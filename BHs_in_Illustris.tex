%% LyX 2.1.2 created this file.  For more info, see http://www.lyx.org/.
%% Do not edit unless you really know what you are doing.
\documentclass[usenatbib]{emulateapj}
\usepackage[latin9]{inputenc}
\setcounter{secnumdepth}{3}
\setcounter{tocdepth}{3}
\usepackage{color}
\usepackage{amsmath}
\usepackage{graphicx}
\usepackage[unicode=true,pdfusetitle,
 bookmarks=true,bookmarksnumbered=false,bookmarksopen=false,
 breaklinks=true,pdfborder={0 0 0},backref=section,colorlinks=true]
 {hyperref}
\hypersetup{
 linkcolor=blue,citecolor=blue}
\usepackage{breakurl}

\makeatletter
%%%%%%%%%%%%%%%%%%%%%%%%%%%%%% User specified LaTeX commands.
\usepackage[caption=false]{subfig}

\makeatother

\begin{document}
\title{Probing the Fundamental Plane of Black Hole Activity in the Illustris Simulation}
\author{Diego F. Gonz\'alez-Casanova, Tim Haines, Zachary Pace, Brianna Smart, Andrea Vang}
\address{Astronomy Department, University of Wisconsin-Madison, 475 North Charter Street, Madison, WI 53706-1582, USA}
\begin{abstract}
Here is our abstract.
\end{abstract}
\keywords{black hole physics -- accretion -- cosmology: miscellaneous}
\maketitle


\section{Introduction}

Accreting black holes such as active galactic nuclei (AGN) are believed
to be important in the evolution of galaxies as the feedback from
AGN can trigger \textbf{I don't believe this. Do we have a citation?}
and/or quench star formation\citep[see ][and references therein]{hopkins2008acosmological}.
Thus, in order to understand galaxy evolution, it is important to
understand the physics of black holes. Specifically, it is important
to probe their influence on the surrounding environment.\textbf{This
is a good start, but why do we care about BHs at all? Talk about angular
momentum catastrophe and how feedback attempts to solve this problem.
Discuss other observational and theoretical evidence for the need
of some feedback mechanism and why BHs do the job (see Hopkins08,Springel05,Schawinksi+ALL)}

There are distinctive signatures of BH activity that can be used to
probe for the presence of a black hole. For example, relativistic
jets emitting synchrotron radiation in the radio band. Strong X-ray
emission from from inverse-Compton scattering in the corona can be
related to the accretion flow of the BH. \citet[hereafter, M03]{merloni2003afundamental}
investigate the properties of $\sim$100 AGN's compact emission in
the X-ray and radio bands and showed that the radio luminosity is
correlated with both the mass and the X-ray luminosity at a highly
significant level. These sources defined a fundamental plane in the
three-dimensional space. The fundamental plane (FP) suggests that
the physical processes regulating the conversion of an accretion flow
into radiative energy could be universal across the entire black hole
mass scale M03.

In our project, we aim to reproduce the fundamental plane of black
hole activity in the Illustris Project to see how well the BHs in
the simulation fit with observations. Most specifically, we aim to
determine the efficiency coefficient $q$ in Equation \ref{LxFP}
in the Illustris simulation using the mass and mass accretion rates
of BHs. From this analysis, we can begin to understand to how well
the Illustris simulates the real physics of black holes in the universe.

This paper is structured as follows. In Section \ref{sec:dis}, we
discuss our results in accordance to the Illustris simulations and
the fundamental plane of BH. In Section~\ref{sec:discussion} we
give our conclusions.


\section{The Illustris Simulation}

The Illustris Project is a series of large-scale cosmological hydrodynamical
simulations of galaxy formation \citep{vogelsberger2014properties}.
The simulation consists of large cosmological situations in a periodic
box with $106.5\;{\rm Mpc}$, simulated with different physics at
different resolutions. It assumes a standard flat $\Lambda$CDM cosmology
with $\Omega_{m,0}=0.2726$, $\Omega_{\Lambda,0}=0.72746$, $\Omega_{b,0}=0.0456$,
and $H_{0}=70.4$ km s$^{-1}$ Mpc$^{-1}$ from the Wilkinson Microwave
Anisotropy Probe 9-year data release \citep{hinshaw2013nineyear}.

In the Illustris simulations, collisionless black hole particles with
a seed mass of $1.42\times10^{5}M_{\odot}$ are placed in dark matter
halos of mass greater than $7.1\times10^{10}M_{\odot}$\citep{sijacki2014theillustris}.
The black hole seeds are allowed to grow through gas accretion or
through mergers with other black holes. At $z=0$, there are 32,542
black holes in total with 3965 black holes more massive than $10^{7}M_{\odot}$.


\section{Sample}

The BH population in Illustris was analysed using the low resolution
simulation at a redshift of $z=0$. The Illustris simulation gives
the mass and accretion rate for each BH. We first eliminate all BH
particles with $M=0$ or $\dot{{\rm M}}=0$, which we assume to be
unphysical. We will hereafter refer to the remaining population (admittedly
somewhat anomalously) as the Full Illustris-2 Black Hole Sample. Analyzing
those two properties for the whole sample set we found there are some
BH with really low accretion rates (Figure \ref{fig:bhpop_full}).
Since we are attempting to fit a linear relationship effectively between
mass an accretion rate, such anomalously low accretion rate black
holes should be omitted.

We accomplished this by first noting that the main sequence of black
holes follows roughly a log-normal distribution in accretion rate
space, centered around approximately $\dot{M}_{BH}=10^{-12.5}M_{\odot}/s$.
However, there also exists a small, secondary population centered
around approximately $\dot{M}_{BH}=10^{-15.7}M\odot/s$ (see Figure
\ref{fig:bhpop_mdot}). To effectively eliminate the low accretion
rate BHs, we impose a hard lower limit on accretion rates, and eliminate
all data below that cutoff.
\begin{figure}
\includegraphics[clip,scale=0.6]{Figures/Illustris2_bhpop_full} \protect\caption{\label{fig:bhpop_full}Each light blue dot represents a single black
hole. Two populations are apparent: a quiescent group and a much larger
accreting population.}
\end{figure}
\begin{figure}
\includegraphics[clip,scale=0.6]{Figures/Illustris2_bhpop_mdot} \protect\caption{\label{fig:bhpop_mdot}The Full Sample exhibits a weak, but noticeable
bimodality in accretion rate space. We fit two log-normal distributions
to the raw distribution, and impose a strict cutoff on the accretion
rate, eliminating most of the anomalously low-accreting population.}
\end{figure}



\section{Analysis}

\label{sec:dis}With the low accretion rate black holes removed from
the sample, we construct a power-law relationship between $M$ and
$\dot{M}$ using linear least-squares (Figure \ref{fig:bhpop_hist2d}).
The best-fit is given by

\begin{equation}
log_{10}(\dot{M})=0.569log_{10}(M)-17.855\;.\label{eq:int_relation}
\end{equation}
This relationship reflects the intrinsic properties of the simulation,
and is subject to no additional models.

The fundamental plane of black holes in the local universe from M03
is shown in Figure \ref{fig:Fp} and defined as

\begin{equation}
\log L_{R}=0.6\log L_{x}+0.78\log M+7.33
\end{equation}
where the $L_{R}$ is the radio luminosity, $L_{X}$ is the X-ray
luminosity, and $M$ is the mass of the black hole. From this relation,
the accretion-powered X-ray luminosity can be expressed as

\begin{equation}
\log L_{x}=\log M+q\log\dot{m}+K_{2}\;,\label{LxFP}
\end{equation}
where $K_{2}$ is normalization constant. Depending on the accretion
flow model, the efficiency coefficient $q$ ranges from 0.5 (optically
thick thin disk accretion flow) to 2.3 (advection dominated accretion
flow). The most significant aspect of the fundamental plane is that
it is a correlation which we can apply our general knowledge of galactic
BHs to AGNs and vice versa.
\begin{figure}
\centering{}\includegraphics[clip,scale=0.35]{Figures/FP} \protect\caption{\label{fig:Fp}Edge-on view of the fundamental plane from M03 relating
the black hole mass to the radio and X-ray luminosity. Symbols indicate
the type of emission-line galaxy of the host and colors correspond
to the mass of the black hole in units of $\log\left({\rm M}_{\odot}\right)$.
Several well-known galaxies hosting an AGN are listed, as well.}
\end{figure}


The fundamental plane of the BHs relates the mass, X-ray luminosity
and accretion rate. Because the luminosity is not an intrinsic property
of the simulation, it had to be calculated. The calculation of the
luminosity for BH can be approximated by two different models. The
first one takes into account the Eddington luminosity of the BH, which
relates it to the mass, and the second uses the thin disk approximation,
which relates the luminosity to the accretion rate. Both of these
approximations for the luminosity gives the bolometric luminosity.
To get the X-ray luminosity from the bolometric luminosity we use
the \citet{elvis1994atlasof} data sample to calculate a template
. The equation relation from the Elvis template is given by

\begin{equation}
L_{x}=0.1947L_{bol}+1.656\times10^{-15}\;.
\end{equation}


Assuming the BH is only emitting at 10\% off the Eddington luminosity
and by means of the Elvis Template the X-luminosity for the BH is
given by

\begin{equation}
L_{x}=623.04M+1.656\times10^{-15}\;.\label{eq:Lx_propto_m}
\end{equation}
By assuming a thin disk approximation with 10\% efficiency we get
that the X-luminosity for the BH is given by

\begin{equation}
L_{x}=4.64\times10^{19}\dot{M}+1.656\times10^{-15}\;,\label{eq:Lx_propto_mdot}
\end{equation}
for all the cases the masses, accretion rates and luminosities are
measured in $M_{\odot}$, $M_{\odot}s^{-1}$ and $L_{\odot}$. With
equations \ref{eq:Lx_propto_m} and \ref{eq:Lx_propto_mdot} and using
the equation \ref{LxFP}. The fundamental plane equation can be rewritten
in terms of mass, accretion rate, k and q. By means of using the intrinsic
mass to accretion relation, equation \ref{eq:int_relation}, the fundamental
plane can be express with only 3 variables.

\begin{multline}
log_{10}(9.03\times10^{18}\dot{M}+1.656\times10^{-15})+1.115log_{10}(\dot{M})\\
-\frac{e}{d}qlog_{10}(\dot{M})+k=0
\end{multline}
\begin{multline}
log_{10}(623.04+\frac{1.656\times10^{-15}}{M})\\
-0.869qlog_{10}(M)+17.855-k=0
\end{multline}


The parameter $q$ holds the information on the properties of the
BH. Hence the adequate value that fits the simulations had to be found.
Using the Newton-Raphson numerical root-finding method for the two
equations and evaluating throw the whole data sample, $q$ is express
in terms of $k$. We find a high correlation between the values of
$q$ and $k$. Since we hope to reproduce the slope $q$ from M03,
which does not provide typical values for $k$, we must approach this
in a rather roundabout manner. We assume that both approximations
for $L_{x}$ are equally good, and will yield similar results, and
combine \ref{eq:Lx_propto_m} and \ref{eq:Lx_propto_mdot} with the
relationship found between $M$ and $\dot{M}$ given by

\begin{multline*}
log_{10}(a_{1}+\frac{b}{M})-dqlog_{10}(M)+eq\\
-log_{10}(a_{2}\dot{M}+b)+\frac{1}{d}log_{10}(\dot{M})-\frac{e}{d}qlog_{10}(\dot{M})=0\;,
\end{multline*}
where $a_{1}$, $a_{2}$, and $b$ are obtained from converting simulation
units to physical units; and $d$ and $e$ are the slope and the intercept
of the power-law found above. The constants have the following values:
\begin{itemize}
\item $a_{1}=623.04$ 
\item $a_{2}=9.03\times10^{18}$ 
\item $b=1.656\times10^{-15}$ 
\item $d=0.869$ 
\item $e=-17.855$ 
\end{itemize}
A value of $q$ is found for each black hole by inputting values of
$M$ and $\dot{M}$, and solving for $q$ using Newton-Raphson numerical
root-finding. For the distribution of black holes in our Illustris-2
sample, the distribution of $q$ values is shown in \ref{fig:q_nr_hist}.
The distribution of $q$ values obtained is strongly peaked at around
$.068$, with a mean of $.0693$.
\begin{figure}
\centering{}\includegraphics[clip,scale=0.15]{Figures/elvis_template}
\protect\caption{\label{fig:Elvis_template}The data points from the BH luminosities
over plot with the best-fit linear relation between $L_{x}$ and $L_{bol}$
for the sample.}
\end{figure}
\begin{figure}
\includegraphics[scale=0.45]{Figures/q_nr_hsit}

\protect\caption{\label{fig:q_nr_hist}A log-histogram of the number of best-fit $q$
values, with one value found per Illustris-2 black hole, using Newton-Raphson
root finding. The maximum likelihood is (SOMEWHERE), with (SOME NUMBER
OF) black holes, and the mean value is $0.693$.}
\end{figure}



\section{Discussion}

\label{sec:discussion}It is well known that the accretion processes
of black holes and the resultant energy deposition into the host galaxy
are necessary ingredients in the evolution of galaxies over cosmic
time. Although we do not understand the causal relationship between
the black hole and its host galaxy, our ability to constrain the mechanisms
responsible are derived principally from simulations of the phenomena.
Using the Illustris-2 simulation, we are able for the first time to
quantify the growth and evolution of supermassive black holes in a
cosmological simulation at the resolution of \textbf{XXX }${\rm M_{\odot}}$.
Using these high-resolution simulation results, we have modelled both
the x-ray and radio luminosities as a function of black hole mass
over the full range present in the Illustris-2 simulation.

In Figure \ref{fig:bhpop_hist2d}, we find that the accretion rate
of black holes with masses less than the median mass of $\mathbf{XXX}{\rm M}_{\odot}$
exhibit accretion rates that differ by \textbf{N} orders of magnitude.
Further, \textbf{XXX\%} exhibit accretion rates less than $10^{-10}\,{\rm M_{\odot}s^{-1}}$
indicating that, on the average, a black hole must accrete for $\mathbf{XX}\,{\rm Gyrs}$
to achieve the median mass. These accretion rates are too low to account
for their masses at $z=0$. For example, a typical low-mass ($10^{7}M_{\odot}$)
black hole accretes at approximately $10^{-12}M_{\odot}s^{-1}$ at
$z=0$. At this rate, such a black hole could only yields a mass of
$\sim10^{5}M_{\odot}$ over the simulation time. Although the high-mass
black holes have generally larger accretion rates in excess of $\dot{M}\approx10^{-10}M_{\odot}s^{-1}$,
the same calculation yields masses of $\sim10^{7}M_{\odot}$ compared
to the actual mass of $\sim10^{9}M_{\odot}$ attained over the simulation
time. This implies that the accretion rates for most of the black
holes in the simulation were likely much larger in the past. This
is consistent with the notion that black hole accretion rates are
tied to the gas fraction in galaxies which was larger at higher redshift.
\textbf{Can we find out when the BH was seeded? I want to make a plot
of seed time vs mass to make this more concrete. Is the linear relation
between mass and $\dot{M}$ known either empirically or from theory?}
\begin{figure}
\includegraphics[clip,scale=0.6]{Figures/Illustris2_bhpop_hist2d}
\protect\caption{\label{fig:bhpop_hist2d}The best-fit power-law relationship between
$M$ and $\dot{M}$ for the sample of Illustris-2 BHs, after imposing
an accretion rate threshold. The fit is overlaid onto a histogram
of the same population, and fifty of the samples data points.}
\end{figure}





\section{Conclusions}

\label{sec:conclusions}In this work, we have analyzed the complete
of supermassive black holes from the state-of-the-art cosmological
hydrodynamical simulation Illustris-2. Applying well-known models
of black hole phenomena, we calculated the x-ray and radio luminosities
as a function of black hole mass. Using the empirically-derived relation
between these three quantities, we have shown that the 

\bibliographystyle{apj}
\bibliography{BHs_in_Illustris}

\end{document}

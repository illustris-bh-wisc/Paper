\section{The Illustris Simulation}

\label{sec:illustris}The Illustris Project is a series of large-scale
cosmological hydrodynamical simulations of galaxy formation \citep{vogelsberger2014properties}.
The simulation consists of large cosmological situations in a periodic
box with $106.5\;{\rm Mpc}$, simulated with different physics at
different resolutions. It assumes a standard flat $\Lambda$CDM cosmology
with $\Omega_{m,0}=0.2726$, $\Omega_{\Lambda,0}=0.72746$, $\Omega_{b,0}=0.0456$,
and $H_{0}=70.4$ km s$^{-1}$ Mpc$^{-1}$ from the Wilkinson Microwave
Anisotropy Probe 9-year data release \citep{hinshaw2013nineyear}.

In the Illustris simulations, collisionless black hole particles with
a seed mass of $1.42\times10^{5}M_{\odot}$ are placed in dark matter
halos of mass greater than $7.1\times10^{10}M_{\odot}$\citep{sijacki2014theillustris}.
The black hole seeds are allowed to grow through gas accretion or
through mergers with other black holes. The black hole seeds are allowed
to grow through gas accretion. At $z=0$, there are 32,542 black holes in total with 3965 black holes more massive than
$10^{7}M_{\odot}$. 

\textbf{Add a note on the accretion mechanism by means of the energy
  and momentum balances due the gravitational force}



%, parametrized in terms of Eddington
%limited Bondi-Hoyle-Lyttleton (BHL)-like accretion (\citet{bondi1952accretion,bondihoyle1044}),
%or through mergers with other black holes. BHL is a non-spherical
%accretion flow which develops when a compact object moves relative
%to a uniform gas cloud. The BHL accretion rate on the black hole is
%given by

%\begin{equation}
%\dot{M}_{B}=\frac{4\pi G^{2}M_{BH}^{2}\rho}{(c_{s}^{2}+v^{2})^{3/2}},
%\end{equation}

%where $\rho$ and $c_{s}$ are the density and sound speed of the
%gas, respectively, $\alpha$ is a dimensionless parameter, and $v$
%is the velocity of the black hole relative to the gas. Then, the accretion
%is limited to the Eddinton rate,

%\begin{equation}
%\dot{M}_{Edd}=\frac{4\pi G^{2}M_{BH}m_{p}}{\epsilon_{r}\sigma_{T}c},
%\end{equation}


%where $m_{p}$ is the proton mass, $\sigma_{T}$is the Thompson cross-section and $\epsilon_{r}$is the radiative efficiency. The radiaitve efficiency is related to the radiated bolometric luminosi%ty, $L_{bol}$, and accretion rate, $\dot{M}_{BH}$, by
%\begin{lyxcode}
%\begin{equation}
%\epsilon_{r}=\frac{L_{r}}{\dot{M}_{BH}}
%\end{equation}

%\end{lyxcode}


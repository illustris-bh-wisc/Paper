\section{Introduction}

Since the earliest simulations of the formation and evolution of disk
galaxies, the problem of stability of the gas in the disk has been
a key issue \citep[e.g., ][]{lucy1977anumerical}. One prominent example
of this is the so-called `angular-momentum catastrophe' \citep{navarro1994accretion}
in which the specific angular momentum of simulated disks is smaller
than values observed in real galaxies by an order of magnitude- resulting
in disks with sizes smaller than observations by a similar factor.
Further, the gas in the simulated disks cools too quickly, causing
it to spiral into the disk's center and form a bulge. However, not
all disk galaxies contain a bulge. Clearly, then, the simulations
are not faithfully representing the physical processes governing disk
formation in nature. The currently accepted solution to this problem
is the introduction of one or more feedback mechanisms which deposit
energy into the gas, heating it, and preventing its aggregation in
the disk's center.

One popular mechanism to drive this feedback is the accretion of gas
onto a central black hole. These so-called active galactic nuclei
(AGN) are believed to be important in the evolution of galaxies as
a mechanism for quenching star formation \citep[see ][and references therein]{hopkins2008acosmological}.
The presence of AGN in $\sim3\%$ of local galaxies at $z\lesssim0.7$
\citep{haggard2010thefield} with an increasing fraction at higher
redshifts \citep{martini2013thecluster} indicates that this mechanism
may contribute a non-negligible source of energy for evolving galaxies.
Additionally, the invocation of an AGN feedback mechanism has been
shown to be necessary in order to recreate the present-day color-magnitude
relation of massive red galaxies \citep{springel2005blackholes}.
Thus, in order to understand galaxy evolution, it is important to
understand the physics governing black hole accretion.

The question then remains as to how we can tie properties of AGN in
simulations back to observations to further constrain the physics
of this feedback mechanism. Fortunately, there are distinctive signatures
of AGN activity that can be used to probe for the presence of an accreting
black hole. For example, relativistic jets emitting synchrotron radiation
in the radio are a strong indicator of recent AGN activity \textbf{{[}CITATION{]}}.
Additionally, X-ray emission typically in excess $\sim10^{42}\,{\rm ergs}\,{\rm s^{-1}}$
stemming from inverse-Compton scattering of electrons in the corona
around the accretion disk of the black hole provides a clear signpost
of a strong source of ionizing energy. \citet[hereafter, M03]{merloni2003afundamental}
investigate the properties of $\sim100$ local galaxies containing
an AGN with compact emission in both the X-ray and radio and show
that the radio luminosity is well correlated both with the mass of
the central black hole and the galaxy's X-ray luminosity. In that
work, M03 define a fundamental plane combining the two observational
signatures (X-ray and radio luminosities) with the accretion flow
onto the black hole. The existence of the fundamental plane suggests
that the physical processes regulating the conversion of gas accreted
onto the black hole into radiative energy could be universal across
the entire scale of black hole masses.

Clearly, then, an important component of modern galaxy formation simulations
is the correct implementation of AGN feedback. In this work, we test
the accuracy of the AGN feedback mechanism in the state-of-the-art
hydrodynamical cosmological simulation Illustris-2. Specifically,
we aim to determine the fundamental plane of black hole activity using
the mass and accretion rates in the simulation and compare against
those measured in M03. From this analysis, we can begin to understand
how well state-of-the-art simulations replicate the real physics of
black hole accretion and AGN feedback in the universe to better understand
the relationship between AGN and the evolution of their host galaxies.

This paper is structured as follows. In Section \ref{sec:illustris},
we outline the implementation of the Illustris-2 simulations. Section
\ref{sec:sample} discusses our sample of black holes extracted from
Illustris-2. Section \ref{sec:analysis} we determine how the black
holes in the Illustris-2 simulation coincide with the fundamental
plane of M03. In Section~\ref{sec:discussion} we discuss our results,
and Section \ref{sec:conclusions} presents our final conclusions.
